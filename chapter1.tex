\chapter{Einleitung}
Schon längst besteht Autofahren aus mehr als lediglich dem Bedürfnis, von A nach B zu kommen.
Ein Fahrer will im Auto vielmehr auch die Möglichkeit haben Dinge zu nutzen, welche nichts mit der Hauptaufgabe des Fahrens (Primäre Aufgabe) zu tun haben.
Zum Beispiel navigiert sich der Autofahrer zu seinem Ziel, hört Musik, findet die nächste Tankstelle in der Umgebung, nimmt ein Telefongespräch an, sowie vieles mehr.
Somit werden in aktuellen Autos die Funktionalität von Informationssystemen im Auto (IVIS von "`In-Vehicle-Information System"') komplexer und die Inhalte umfangreicher \citep{Kern:2009}.
Hierbei besteht die Aufgabe und Herausforderung für Designer diese Masse an Information in geeigneter Weise für den Fahrer darzustellen und nutzbar zu machen.
Dauert eine Interaktion mit einem IVIS zu lange oder ist sie schwer zu verstehen sind dies häufig Gründe für zu große Ablenkungen beim Fahren, was Autounfälle mit sich zieht \citep{neale2005overview}.
Daher sollten diese potenziellen Ablenkungen so minimal wie möglich gehalten werden.
Zusätzlich ist es wichtig, dem Nutzer eine einfache und intuitive Eingabe des Informationssystems im Auto zu ermöglichen, ohne ihn gleichzeitig zu sehr visuell oder mental zu belasten. 

Hinzu kommt, dass die Bedienung von jeglichen Funktionen im Auto längst nicht mehr unimodal sind.
Unimodal bezeichnit hierbei die Interaktion zwischen Mensch und Auto, welche lediglich eine Modalität verwendet, wie zum Beispiel Sprache.
Dagegen is es dem Fahrer möglich verschiedenste Funktionen multimodal auszuführen.
Zu den üblichen haptischen Bedienelementen kommen somit zusätzlich Touchdisplays oder Möglichkeiten der Sprachbedienung. 
Allerdings unterstützen aktuell Nutzungsschnittstellen im Autos nur einige Varianten von unimodaler Interaktion. 
Ein weit verbreiteter Modus ist die Spracheingabe, die zum Beispiel oft genutzt wird, um ein Telefonanruf zu starten oder anzunehmen. 

Ein wesentlicher Vorteil der Sprachsteuerung ist, dass die Hand nicht vom Lenkrad genommen werden muss und auch der Blick auf der Straße verbleiben kann.
\citet{maciej2009comparison} sind der Überzeugung, dass Sprachbedienung ein Unerlässlichkeit in zukünftigen Autos sein wird.
Allerdings ist bei einer Autofahrt mit mehreren Mitfahrern, bei der geredet und Musik gehört wird, die Sprachsteuerung ungeeignet. 

Auch Gestensteuerungen sind in einigen Autos bereits vertreten.
Zum Beispiel wird im BMW 7er eine Kreisbewegung mit dem Finger nach rechts bzw. links erkannt, um die Lautstärke zu erhöhen beziehungsweise zu verringern. 
Es ist allerdings meist nicht möglich mit einer dieser neuen Modalitäten alle vorhandenen Funktionen auszuführen.
Sobald allerdings ein Interface seriell-redundant ist, das heisst TODO(erklaerung von serial redundant), können auch Modalitäten wie zum Beispiel Sprache, Gestik oder Touch beliebig gewechselt werden \citep{neuss_2001}. 

Mit einem seriell-redundanten multimodalen Informationssystemen im Auto könnte der Fahrer selbst entscheiden in welchen Situationen welcher Modus für den jeweiligen Schritt am einfachsten und sichersten ist.
Dies ist stellt nach \citet{Muller_2011} einen wesentlichen Vorteil im Auto dar.
Natürlich gibt es auch Kombinationen, welche mit bestimmten Modalitäten weniger sinnvoll sind als andere.
Unsere Motivation ist deshalb solche multimodalen Interaktionen in IVIS zukünftig besser verstehen um diese optimal umsetzen zu können.
Es ist wichtig die schnellsten und einfachsten Kombinationen verschiedener Modalitäten zu kennen, um dem Fahrer auch die besten Möglichkeiten in verständlicher Weise anzuzeigen.
Damit soll die visuelle und mentale Beanspruchung für den Fahrer so gering wie möglich gehalten werden. 

Um Interfaces bereits in einem frühen Stadium der Entwicklung zu testen wurde in der Vergangenheit häufig das Keystroke-Level Modell in der ursprünglichen oder den erweiterten Varianten verwendet.
Damit können bereits vor der Implementierung eines Prototyps Interfacekonzepte auf ihre Bediendauer getestet werden, indem die Dauer einer gewählten Aufgabe vorhergesagt wird.
Somit wird es möglich, auf einfache Art verschiedene Varianten zu vergleichen.
Das Keystroke-Level Modell bezieht sich in der ursprünglichen Variante nur auf Desktop-basierte Anwendungen, die sich hauptsächlich auf Textverarbeitungsprogramme bezogen.
In den letzten Jahrzehnten wurde dieses vereinfachte Konzept jedoch auch auf Geräte und Anwendungen übertragen und erwies sich hierbei als geeignete Methode zur Einschätzung der Interaktionsdauer. 

Das Ziel dieser Arbeit ist es, dieses Konzept für seriell redundante multimodale IVIS zu adaptieren und anzupassen.
Dazu entwickeln wir ein Modell, mit dem Interaktionszeiten eines multimodalen IVIS vorhergesagt und verglichen werden können.
Wir konzentrieren uns auf die Modalitäten Touch, Sprache und Geste und auf die entstehenden Kosten bei einem Wechsel zwischen zwei Modalitäten.
Jede Modalität besitzt ihre Vorteile und Nachteile und können geeigneter Kombination den Fahrer optimal unterstützen.

Im Zuge dieser Masterarbeit soll daher die Bedienung im Fahrzeug beobachtet werden, mit dem Ziel die multimodale Interaktion für künftige Systeme modellieren zu können.
Ein besonderer Aspekt liegt dabei auf der multimodale Bedienung, bei der zum Beispiel Handlungen nacheinander oder alternativ über verschiedene Modalitäten abgewickelt werden.
Die Aktionen können dementsprechend unterschiedlich lang dauern und Wechselkosten enthalten.
Basierend auf den empirisch zu ermittelnden Operatoren und Operatorzeiten soll dann ein erweitertes Modell erstellt werden, welches die Vorhersage für eine multimodale Interaktion im Fahrzeug ermöglicht.
Zu diesem Zweck ist es sowohl unter den Aspekten der Bedienbarkeit und User Experience also auch der Fahrsicherheit und Fahrerablenkung geplant, verschiedene Untersuchungen (zum Beispiel Umfragen, Fokusgruppen, Laborstudien) durchzuführen.
Hierbei sollen Operatoren und Interaktionszeiten gemessen und validiert werden. 

\section*{Gliederung}
Im Zuge dieser Arbeit wurde folgendermaßen vorgegangen.
Zu Beginn werden in \textbf{Kapitel \ref{cha:verwandteArbeiten}} die Grundlagen und Richtlinien zu Informationssystemen im Auto, deren Design und Evaluationsmöglichkeiten erläutert.
Anschließend wird das ursprüngliche Keystroke-Level Modell und deren Erweiterungen auf tragbare Geräte, für Touch und dem KLM im automobilen Kontext erläutert.
Außerdem befassen wir uns mit multimodalen Interaktionen im generellen, sowie mit multimodalen Interaktionen im automobilen Kontext. 

In \textbf{Kapitel \ref{cha:Workshop}} erhalten wir in einer Brainstorming Runde einen Überblick über multimodale Interaktionen im Auto und deren bereits vorhandenen und möglichen Umsetzungsvarianten.
Dort sammeln wir gemeinsam Ideen und gruppieren sie Schritt für Schritt bis wir auf deren Grundlage unser Konzept aufbauen können.
Aus den gesammelten Ideen und Ergebnissen leiten wir unsere Aktionen her und konstruieren geeignete Anwendungsbeispiele. 

\textbf{Kapitel \ref{cha:Prototyp}} beschreibt die Idee und Implementierung des multimodalen Prototyps, der per Touch, Geste und Sprache, sowie je aus einer Kombination aus zwei Modalitäten bedient werden soll. 
Wir gehen auf die Anwendungsbeispiele und die Umsetzungen ein. 

Das Studiendesign, die Durchführung der Studie und die anschließende Auswertung wird in \textbf{Kapitel \ref{cha:Studie}} erläutert.
Die 22 Probanden führen vier verschiedene Anwendungsbeispiele mit allen Moduskombinationen durch.
Dies resultiert in 33 verschiedene Kombinationen, welche von jedem Probanden durchgeführt werden, um für alle Aktionen unseres Modells genügend Daten für jede Moduskombination und deren Wechselkosten zu erhalten.
Aus den Studienergebnissen werden die Zeiten der Aktionen erhoben und ein Modell mit Durchschnittszeiten erstellt.
Des weiteren untersuchen wir, welchen Einfluss ein Moduswechsel auf die Interaktionszeiten hat und in welchem Maße.
Anschließend werden die Ergebnisse dieser Studie diskutiert und mit anderen Arbeiten verglichen.

Um die ermittelten Aktionszeiten des erstellten Modells zur Vorhersage/Abschätzung von Interaktionszeiten zu validieren, wird in \textbf{Kapitel \ref{cha:Evaluation}} der Prototyp angepasst und in einer zweiten Studie an zehn weiteren Probanden getestet.
Dazu werden die Daten ausgewertet und mit unserem erstellen Modell verglichen.
Anschließend diskutieren wir über die Erkenntnisse.

\textbf{Kapitel \ref{cha:Zusamenfassung}} fasst die Erkenntnisse beider Studien und der gesamten Arbeit zusammen und gibt einen Ausblick auf weitere Forschungsideen.
