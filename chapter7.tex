\chapter{Zusammenfassung und Ausblick}\label{cha:Zusamenfassung}
Motiviert durch die Vorteile verschiedener Modalit�ten und die damit verbundene M�glichkeit den Nutzer zu entlasten und ihn somit weniger von der Fahraufgabe abzulenken. Der Fahrer kann selbst entscheiden in welchen Situationen welcher Modus f�r den jeweiligen Schritt am geeignetsten und ungef�hrlichsten ist, was auch laut \citet{Muller_2011} einen wesentlichen Vorteil in Autos darstellt.

Deshalb war unsere Motivation solche multimodalen Interaktionen in IVIS besser verstehen zu k�nnen und diese optimal umsetzen zu k�nnen. Mit der Entwicklung unsres Modells zur Vorhersage von multimodalen Interaktionszeiten ist es uns gelungen Designer die M�glichkeit zu geben bereits in einem fr�hem Stadium der Entwicklungsphase multimodale Interaktionen einzusch�tzen und zu vergleichen. Damit k�nnen Designer in geeigneter Weise die besten potenziellen Varianten unterst�tzen, um dem Fahrer auch die besten M�glichkeiten in verst�ndlicher Weise anzuzeigen. Damit soll nat�rlich auch die visuelle und mentale Beanspruchung f�r den Fahrer so gering wie m�glich gehalten werden. 

Ob eine Variante besonders gut oder schlecht ist h�ngt nat�rlich von der Situation ab, Deshalb w�re ein seriell redundantes multimodalen IVIS am besten, um in jedem Schritt den Fahrer selbst die beste Modalit�t w�hlen zu lassen. Die Eingabe eines Zieles per Touch ist zum Beispiel im stehenden Auto eine gute Variante, die jedoch w�hrend der Fahrt den Fahrer durch die l�ngere Interaktion zu sehr ablenken k�nnte. In diesem Fall stellt die Texteingabe per Sprache eine sehr geeignete Alternative dar.

Unser Modell zur Vorhersage von multimodalen Interaktionen im IVIS sollte in Zukunft noch dahingehend weiterentwickelt werden auch haptische Bedienelemente miteinzubeziehen. Zudem kann das Modell auf weitere Gesten erweitert werden und auch die weitere Untersuchungen von gesprochenen S�tzen k�nnte hilfreich sein.

Im Zuge dieser Masterarbeit wurde im Workshop Connected Minds die Bedienung im Fahrzeug beobachtet und diskutiert. Es wurde ein Konzept f�r ein multimodales Modell entworfen und in einem multimodalen Prototypen umgesetzt. Dieser kann mit Touch, Sprache und Geste bedient werden. Zur Erhebung der Interaktionszeiten und der Evaluierung des Modells wurde der Prototyp seriell exklusiv umgesetzt, da dem Nutzer die Modalit�t vorgegeben war. Unser Modell unterscheidet verschiedene Aktionen in Abh�ngigkeit der Modalit�t und enth�lt entstehende Wechselkosten wenn der Nutzer die Modalit�t �ndert. 